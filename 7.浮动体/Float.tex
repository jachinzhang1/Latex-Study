\documentclass{ctexart}
%注释快捷键为ctrl+T

%\usepackage{ctex}
\usepackage{graphicx}
\graphicspath{{figures/}}

%标题控制(caption, bicaption等宏包)
%并排与子图表(subcaption, subfig, floatrow等宏包)
%绕排(picinpar, wrapfig等宏包)

\begin{document}
	在\LaTeX{}文档中加入的插图JoJo见图\ref{fig-jojo}。%用\ref{}实现标签的交叉引用
	
	\begin{figure}[htbp]
		%htbp:通过可选参数指定浮动体排版位置实现灵活分页(htbp为允许各个位置)
		%	h:here	代码所在的上下文位置
		%	t:top	代码所在页面或者之后页面的顶部
		%	b:bottom代码所在页面或者之后页面的底部
		%	p:page	独立一页(浮动页面)
		\centering%让环境中的内容居中排版
		\includegraphics{Jotaro}
		\caption{jotaro}%设置插图标题
		\label{fig-jojo}%设置标签
	\end{figure}

	迪奥对此表示了肯定(图\ref{fig-dio})。
	\begin{figure}[htbp]
		\centering
		\includegraphics{dio}
		\caption{dio}\label{fig-dio}
		%label等效于图表的序号,但引用标签可以实现方便的排序。
	\end{figure}
	
	在\LaTeX 中还可以呈现表\ref{tab-score}所示的表格。
	\begin{table}[h]
		\centering
		\caption{成绩单}\label{tab-score}
		\begin{tabular}{l || c | c | c | p{1.5cm}}% | 产生表格竖线(||双竖线)
			%生成表格环境
			%此处生成5列表格(l=左对齐,c=居中对齐,r=右对齐)
			%p可以规定列宽,内容超过宽度自动换行
			\hline%产生表格横线
			姓名 & 语文 & 数学 & 外语 & 备注 \\
			\hline\hline%产生双横线
			张三 & 87 & 100 & 93 & 优秀 \\
			\hline
			李四 & 75 & 64 & 52 & 补考另行通知 \\
			\hline
			王五 & 80 & 82 & 78 & \\
			\hline
		\end{tabular}
	\end{table}
	
\end{document}