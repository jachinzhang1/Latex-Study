%latex中字体有五种属性:
%字体编码:
%	正文字体编码OT1, T1, EU1等
%	数学字体编码OML, OMS, OMX等
%字体族:
%	罗马字体:笔画起始处有装饰
%	无衬线字体:笔画起始处无装饰
%	打字机字体:每个字符宽度相同,又称等宽字体
%字体系列:粗细、宽度
%字体形状:直立、斜体、伪斜体、小型大写
%字体大小

\documentclass[10pt]{article}
%方括号调默认字体的大小(一般只有10,11,12pt)
\usepackage{ctex}

\newcommand{\myfont}{\textbf{\textsf{\textit{Fancy Text}}}}
%用newcommand定义新命令

\begin{document}
	%字体族设置(罗马字体、无衬线字体、打字机字体)
	\textrm{Roman Family}
	
	\textsf{Sans Serif Family}
	
	\texttt{Typewriter Family}
	
	{\rmfamily Roman } 	{\sffamily SansSerif }  	{\ttfamily Typewriter }
	
	{\rmfamily What do you want?}
	
	{\sffamily I want an apple!}
	
	{\ttfamily I want a banana!}
	
	%字体系列(粗细、宽度)
	\textmd{Medium Series}	
	\textbf{Boldface Series}
	
	{\mdseries Medium}	{\bfseries Boldface}
	
	%字体形状(直立、斜体、伪斜体、小型大写
	\textup{Upright Shape}	\textit{Italic Shape}
	\textsl{Slanted Shape}	\textsc{Small Caps Shape}
	
	{\upshape Upright}  {\itshape Italic}	{\slshape Slanted} {\scshape SmallCaps}
	
	%中文字体
	{\songti 宋体} \quad {\heiti 黑体} \quad {\fangsong 仿宋} \quad {\kaishu 楷书}
	%\quad代表空格
	
	中文字体的\textbf{粗体}用黑体表示,\textit{斜体}用楷书表示。
	
	%字体大小
	{\tiny 			hello}\\%双反斜杠 无段首空格换行
	{\scriptsize 	hello}\\
	{\footnotesize  hello}\\
	{\small 		hello}\\
	{\normalsize 	hello}\\
	{\large 		hello}\\
	{\LARGE 		hello}\\
	{\huge 			hello}\\
	{\Huge		 	hello}\\
	
	%中文字号大小设置详可参阅ctex宏集
	\zihao{-1} 你好!
	
	\myfont
\end{document}